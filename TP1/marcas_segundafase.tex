- Provérbios turcos/búlgaros/chineses/holandeses                DONE
*{espaco} ou * &quot; ou *&quot; ou *{sem espaco}

\n ou &quot;

- Provérbios holandeses                                         DONE
*{espaco} ou *{espacos}

\n ou &quot;


-------------------------


- Provérbios venezuelanos/poloneses                                       DONE
*&quot; ou *{sem espaco,logo letras} ou * &quot; ou *{espaco}{letras}

&quot; ou \n


-------------------------


- Provérbios russos                                                      DONE
*{espaco}

\n


-------------------------


- Provérbios árabes
*&quot; ou *{espaco} ou *{sem espaco}

&quot; ou \n


-------------------------


- Provérbios japoneses/alemães/latinos                                   DONE ("* &quot;" no latinos ?????)
*{espaco} ou *{sem espaco} ou * '' ou *'' ou *'''

\n ou ''\n

- Provérbios galegos:                                                    DONE (algumas merdinhas) está junto com o de cima
* ''

'' ou &quot;.'' ou '''


-------------------------


- Provérbios italianos                                                   DONE (coisas erradas no inicio das frases)
* ''&quot; ou * ''

&quot; ou ''\n


-------------------------


- Provérbios espanhóis                                                   DONE
*{espaco} ou * &quot; ou * ''

\n ou &quot; ou ''\n


-------------------------


- Provérbios israelenses                                                 DONE
*{espaco} ou * '''

\n ou '''\n


-------------------------


- Provérbios córsicos/egípcios                                           DONE
* ''' ou :''' ou : ''' ou '''

'''                  \n tambem dá mas não é indicado(????)


-------------------------


- Provérbios franceses                                                   DONE
* &quot; ou (inicio de linha)      não sei se vai funcionar

&quot; ou \n


-------------------------


- Provérbios tunisianos                                                   DONE

* “''A diferença entre o deserto'' e ''meu jardim'' não é a água,&lt;ref&gt;JUSTO, H.S. Diferenciação: o caminho da vantagem competitiva.Porto Alegre: Evangraf. 2009.p.81.&lt;/ref&gt; mas o homem”.
