\documentclass[11pt,a4paper]{report}

\usepackage[utf8]{inputenc}
\usepackage[portuges]{babel}
\usepackage{indentfirst}
\usepackage{graphicx}
\usepackage{float}
\usepackage{caption}
\usepackage{subcaption}
\usepackage[T1]{fontenc}
\usepackage{listings}
\usepackage{amsmath}
\usepackage{mathtools}
\renewcommand{\familydefault}{\sfdefault}

% packages que adicionei do stor
\usepackage{xspace}
\setlength{\oddsidemargin}{-1cm}
\setlength{\textwidth}{18cm}
\setlength{\headsep}{-1cm}
\setlength{\textheight}{23cm}

\title{Processamento de Linguagens (3º ano de Curso)\\
	\textbf{Trabalho Prático Nº1}\\ Relatório de Desenvolvimento}
\author{Diogo Braga\\ A82547 \and João Silva\\ A82005 \and Ricardo Caçador\\ A81064}
\date{\today}

\begin{document}

\maketitle

\begin{abstract}
	Neste relatório é apresentada a resolução de um exercício referente ao TP1, que tem como objetivos a utilização de Expressões Regulares para descrição de padrões de frases, e a utilização do Flex para gerar filtros de texto em C.
Outro objetivo será ainda desenvolver, apartir de ERs, sistemática e automaticamente Processadores de Linguagens Regulares, que filtrem ou transformem textos com base no conceito de regras de produção Condição-Ação.
\end{abstract}

\tableofcontents

\newpage


\chapter{Introdução}
\label{sec:intro}

Seguindo a fórmula \emph{exe= (N\_Alu\% 7) + 1} e o número de aluno mais baixo presente no nosso grupo (81064), o enunciado correspondente é o \textbf{5 WikiQuotes: provérbios}.

\chapter{Conclusão}
\label{sec:conclusao}




\end{document}
