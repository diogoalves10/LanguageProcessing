\documentclass[11pt,a4paper]{report}

\usepackage[utf8]{inputenc}
\usepackage[portuges]{babel}
\usepackage{indentfirst}
\usepackage{graphicx}
\usepackage{float}
\usepackage{caption}
\usepackage{subcaption}
\usepackage[T1]{fontenc}
\usepackage{listings}
\usepackage{amsmath}
\usepackage{mathtools}
\usepackage{tikz}
\renewcommand{\familydefault}{\sfdefault}

% packages que adicionei do stor
\usepackage{xspace}
\setlength{\oddsidemargin}{-1cm}
\setlength{\textwidth}{18cm}
\setlength{\headsep}{-1cm}
\setlength{\textheight}{23cm}

\title{Processamento de Linguagens (3º ano de Curso)\\
	\textbf{Trabalho Prático Nº3 (YACC)}\\ Relatório de Desenvolvimento}
\author{Diogo Braga\\ A82547 \and João Silva\\ A82005 \and Ricardo Caçador\\ A81064}
\date{\today}

\begin{document}

\maketitle

\begin{abstract}
	Neste relatório é apresentada a resolução de um exercício referente ao TP3, que tem como principais objetivos:
	\begin{itemize}
		\item aumentar a experiência de uso do ambiente \textbf{Linux}, da linguagem imperativa  \textbf{C}, e de algumas ferramentas de apoio à programação;
 		\item rever e aumentar a capacidade de escrever \textit{gramáticas independentes de contexto (GIC)}, que satisfaçam a condição LR(), para criar Linguagens de Domínio Específico (DSL);
 		\item desenvolver processadores de linguagens segundo o método da \textit{tradução dirigida pela sintaxe}, suportado numa \textit{gramática tradutora (GT)};
 		\item utilizar geradores de compiladores como o par \textbf{flex/yacc}.
	\end{itemize}
\end{abstract}

\tableofcontents

\newpage

\chapter{Introdução}
\label{chap:intro}

Seguindo a fórmula \emph{exercício = (N\_Alu\% 6)  +  1} e o número de aluno mais alto presente no nosso grupo (82547), o enunciado correspondente é o \textbf{6  -  Construtor de diaporama}.

Este enunciado apresentou-nos um tipo de apresentações visuais suportadas por um conjunto de diapositivos ou acetatos/slides que é conhecido como \textbf{Diaporama}. O \textbf{diaporama} é uma sequência de elementos. Cada elemento pode ser/conter:
\begin{itemize}
	\item Uma página inicial de abertura e créditos;
	\item imagens;
	\item páginas simples (exemplo:  uma lista de items);
	\item video;
	\item opcionalmente, título;
	\item opcionalmente, audio.
\end{itemize}

Tendo estes aspetos em conta é requerido que se crie uma sequência de páginas HTML temporizadas, em que cada página HTML corresponderá a um diapositivo.

Com este relatório pretendemos apresentar as nossas opções e algoritmos desenvolvidos para a realização do gerador de diaporamas. Pretendemos também apoiar aquelas que foram as nossas soluções, com conhecimento obtido nas aulas teóricas.

Para uma melhor visão do que irá ser abordado neste relatório deixamos uma breve descrição daquilo que foi feito. No segundo capítulo foi feita uma análise informal e uma especificação dos requisitos deste projeto. No terceiro capítulo foi realizado o desenho da conceção no qual estão envolvidos os algoritmos e estruturas de dados usados. No quarto capítulo mostramos alguns exemplos de implementações e vários resultados de testes realizados. No quinto capítulo apresentamos as funcionalidades estendidas do nosso projeto. Por último no capítulo 6 fazemos uma retrospetiva do trabalho realizado e concluímos.


\chapter{Análise e Especificação}
\label{chap:analise}

Analisando o problema como um todo, a fase de análise foi bastante simples e curta. Bastou apenas seguir o enunciado para começar a pensar o que seria um \textbf{diaporama}, que elementos poderia conter, que tipo de slides poderia apresentar bem como a disposição do conteúdo a apresentar.

No final desta supervisão do conceito de \textbf{diaporama} apuramos todos os elementos que este poderia conter, que são nomeadamente:

\begin{itemize}
	\item Um nome da apresentação;
	\item Diapositivos, cada um contendo:
	\begin{itemize}
		\item Tempo de disposição;
		\item Body, onde reside conteúdo, que pode ser/conter de diversos tipos, nomeadamente:
		\begin{itemize}
			\item Uma página inicial de abertura e créditos;
			\item Imagens;
			\item Páginas simples (exemplo:  uma lista de items);
			\item Vídeo;
			\item Título;
			\item Áudio.
		\end{itemize}
	\end{itemize}
\end{itemize}

De referir ainda que cada tipo de diapositivos é aprofundado contendo variados parâmetros que serão apresentados no próximo capítulo.

Por último, após a análise dum diaporama, foi de grande relevância a investigação por conteúdo \textbf{HTML} que pudesse representar numa página cada tipo de diapositivo anteriormente definido.

\chapter{Conceção/desenho da Resolução}
\label{chap:concecao}

Neste capítulo baseamo-nos nas análises anteriormente efetuadas que serviram de base inicial para a construção duma \textbf{gramática independente de contexto (GIC)} que representasse um diaporama e consequentemente para a construção duma \textbf{gramática tradutora (GT)} que despontasse determinadas ações consoante o momento de parsing em que se encontrava.


\section{Gramática Independente de Contexto}

Tendo em conta a análise realizada construímos uma \textit{Gramática Independente de Contexto} \textbf{G = (T, NT , S, P)}, em que \textbf{T} é conjunto de terminais, \textbf{NT} é o conjunto de não terminais, \textbf{S} é o axioma e \textbf{P} é o conjunto de produções. A gramática \textbf{G} pode então ser definida por:

\vspace{0.5cm}

\textbf{T} = \{ STYLE, DIAPOSITIVO, CRED, IMG, LI, VID, LB, TITL, AUD, ',' , ';' , '\{' , '\}' , '(' , ')' , NUM, STRING\}

\vspace{0.5cm}

\textbf{NT} = \{ Diaporama, Estilo, Nome, Elementos, Elemento, Body, Tipos, Tipo, Credito, Imagem, Item, Video, Opcoes, Titulo, Audio, Tempo, Width, Height \}

\vspace{0.5cm}

\textbf{S} = \{ Diaporama \}

\vspace{0.5cm}

\textbf{P} = \{ Diaporama $\rightarrow$ Nome ',' Elementos 

\vspace{0.2cm}
\hspace{3.15cm} | Nome ',' STYLE Estilo ',' Elementos 

\vspace{0.2cm}
\hspace{1.0cm} Nome $\rightarrow$ STRING

\vspace{0.2cm}
\hspace{1.0cm} Elementos $\rightarrow$ Elemento

\vspace{0.2cm}
\hspace{3.05cm} | Elementos ',' Elemento

\vspace{0.2cm}
\hspace{1.0cm} Elemento $\rightarrow$ DIAPOSITIVO '\{' Tempo ';' Body '\}' 

\vspace{0.2cm}
\hspace{1.0cm} Tempo $\rightarrow$ NUM

\vspace{0.2cm}
\hspace{1.0cm} Body $\rightarrow$ '(' Opcoes ')' ',' Tipos

\vspace{0.2cm}
\hspace{2.3cm} | Tipos

\vspace{0.2cm}
\hspace{1.0cm} Tipos $\rightarrow$ Tipo

\vspace{0.2cm}
\hspace{2.3cm} | Tipos ',' Tipo

\vspace{0.2cm}
\hspace{1.0cm} Tipo $\rightarrow$ Credito

\vspace{0.2cm}
\hspace{2.2cm} | Imagem

\vspace{0.2cm}
\hspace{2.2cm} | Item

\vspace{0.2cm}
\hspace{2.2cm} | Video

\vspace{0.2cm}
\hspace{2.2cm} | LineBreak

\vspace{0.2cm}
\hspace{1.0cm} Credito $\rightarrow$ CRED STRING

\vspace{0.2cm}
\hspace{1.0cm} Imagem $\rightarrow$ IMG STRING

\vspace{0.2cm}
\hspace{2.65cm} | IMG STRING Width Height

\vspace{0.2cm}
\hspace{1.0cm} Width $\rightarrow$ NUM

\vspace{0.2cm}
\hspace{1.0cm} Height $\rightarrow$ NUM

\vspace{0.2cm}
\hspace{1.0cm} Item $\rightarrow$ LI STRING

\vspace{0.2cm}
\hspace{1.0cm} Video $\rightarrow$ VID STRING

\vspace{0.2cm}
\hspace{2.3cm} | VID STRING Width Height 

\vspace{0.2cm}
\hspace{1.0cm} LineBreak $\rightarrow$ LB 

\vspace{0.2cm}
\hspace{1.0cm} Opcoes $\rightarrow$ Opcoes ',' Audio

\vspace{0.2cm}
\hspace{2.6cm} | Opcoes ',' Titulo

\vspace{0.2cm}
\hspace{2.6cm} | Titulo

\vspace{0.2cm}
\hspace{2.6cm} | Audio

\vspace{0.2cm}
\hspace{1.0cm} Titulo $\rightarrow$ TITL STRING 

\vspace{0.2cm}
\hspace{1.0cm} Audio $\rightarrow$ AUD STRING 


\hspace{0.7cm} \}


\section{Gramática Tradutora}

//////////////////// FALTA FAZER

\subsection{Funções em C}

//////////////////// FALTA FAZER

\chapter{Codificação e Testes}
\label{chap:codificacao}

//////////////////// FALTA FAZER

\section{Estruturas de Dados}

//////////////////// FALTA FAZER

\section{Alternativas, Decisões e Problemas de Implementação}

//////////////////// FALTA FAZER

\section{Testes realizados e Resultados}

//////////////////// FALTA FAZER

\chapter{Extras}
\label{chap:extras}

//////////////////// FALTA FAZER

\chapter{Conclusão}
\label{chap:concl}

//////////////////// FALTA FAZER

\end{document}
