
\documentclass[11pt,a4paper]{report}%especifica o tipo de documento que tenciona escrever: carta, artigo, relatório... neste caso é um relatório
% [11pt,a4paper] Define o tamanho principal das letras do documento. caso não especifique uma delas, é assumido 10pt
% a4paper -- Define o tamanho do papel.

\usepackage[portuges]{babel}%Babel -- irá activar automaticamente as regras apropriadas de hifenização para a língua todo o
                                   %-- o texto gerado é automaticamente traduzido para Português.
                                   %  Por exemplo, “chapter” irá passar a “capítulo”, “table of contents” a “conteúdo”.
                                   % portuges -- específica para o Português.
\usepackage[utf8]{inputenc} % define o encoding usado texto fonte (input)--usual "utf8" ou "latin1

\usepackage{graphicx} %permite incluir graficos, tabelas, figuras
\usepackage{url} % para utilizar o comando \url{}
\usepackage{enumerate} %permite escolher, nas listas enumeradas, se os iems sao marcados com letras ou numeros-romanos em vez de numeracao normal

%\usepackage{apalike} % gerar biliografia no estilo 'named' (apalike)

\usepackage{color} % Para escrever em cores

\usepackage{multirow} %tabelas com multilinhas
\usepackage{array} %formatação especial de tabelas em array

\usepackage[pdftex]{hyperref} % transformar as referências internas do seu documento em hiper-ligações.

%Exemplos de fontes -- nao e vulgar mudar o tipo de fonte
%\usepackage{tgbonum} % Fonte de letra: TEX Gyre Bonum
%\usepackage{lmodern} % Fonte de letra: Latin Modern Sans Serif
%\usepackage{helvet}  % Fonte de letra: Helvetica
%\usepackage{charter} % Fonte de letra:Charter

\definecolor{saddlebrown}{rgb}{0.55, 0.27, 0.07} % para definir uma nova cor, neste caso 'saddlebrown'

\usepackage{listings}  % para utilizar blocos de texto verbatim no estilo 'listings'
%paramerização mais vulgar dos blocos LISTING - GENERAL
\lstset{
	basicstyle=\small, %o tamanho das fontes que são usadas para o código
	numbers=left, % onde colocar a numeração da linha
	numberstyle=\tiny, %o tamanho das fontes que são usadas para a numeração da linha
	numbersep=5pt, %distancia entre a numeração da linha e o codigo
	breaklines=true, %define quebra automática de linha
    frame=tB,  % caixa a volta do codigo
	mathescape=true, %habilita o modo matemático
	escapeinside={(*@}{@*)} % se escrever isto  aceita tudo o que esta dentro das marcas e nao altera
}
%
%\lstset{ %
%	language=Java,							% choose the language of the code
%	basicstyle=\ttfamily\footnotesize,		% the size of the fonts that are used for the code
%	keywordstyle=\bfseries,					% set the keyword style
%	%numbers=left,							% where to put the line-numbers
%	numberstyle=\scriptsize,				% the size of the fonts that are used for the line-numbers
%	stepnumber=2,							% the step between two line-numbers. If it's 1 each line
%											% will be numbered
%	numbersep=5pt,							% how far the line-numbers are from the code
%	backgroundcolor=\color{white},			% choose the background color. You must add \usepackage{color}
%	showspaces=false,						% show spaces adding particular underscores
%	showstringspaces=false,					% underline spaces within strings
%	showtabs=false,							% show tabs within strings adding particular underscores
%	frame=none,								% adds a frame around the code
%	%abovecaptionskip=-.8em,
%	%belowcaptionskip=.7em,
%	tabsize=2,								% sets default tabsize to 2 spaces
%	captionpos=b,							% sets the caption-position to bottom
%	breaklines=true,						% sets automatic line breaking
%	breakatwhitespace=false,				% sets if automatic breaks should only happen at whitespace
%	title=\lstname,							% show the filename of files included with \lstinputlisting;
%											% also try caption instead of title
%	escapeinside={\%*}{*)},					% if you want to add a comment within your code
%	morekeywords={*,...}					% if you want to add more keywords to the set
%}

\usepackage{xspace} % deteta se a seguir a palavra tem uma palavra ou um sinal de pontuaçao se tiver uma palavra da espaço, se for um sinal de pontuaçao nao da espaço

\parindent=0pt %espaço a deixar para fazer a  indentação da primeira linha após um parágrafo
\parskip=2pt % espaço entre o parágrafo e o texto anterior

\setlength{\oddsidemargin}{-1cm} %espaço entre o texto e a margem
\setlength{\textwidth}{18cm} %Comprimento do texto na pagina
\setlength{\headsep}{-1cm} %espaço entre o texto e o cabeçalho
\setlength{\textheight}{23cm} %altura do texto na pagina

% comando '\def' usado para definir abreviatura (macros)
% o primeiro argumento é o nome do novo comando e o segundo entre chavetas é o texto original, ou sequência de controle, para que expande
\def\darius{\textsf{Darius}\xspace}
\def\antlr{\texttt{AnTLR}\xspace}
\def\pe{\emph{Publicação Eletrónica}\xspace}
\def\titulo#1{\section{#1}}    %no corpo do documento usa-se na forma '\titulo{MEU TITULO}'
\def\super#1{{\em Supervisor: #1}\\ }
\def\area#1{{\em \'{A}rea: #1}\\[0.2cm]}
\def\resumo{\underline{Resumo}:\\ }

%\input{LPgeneralDefintions} %permite ler de um ficheiro de texto externo mais definições

\title{Programação Imperativa (?º ano de Curso)\\
       \textbf{Trabalho Prático N}\\ Relatório de Desenvolvimento
       } %Titulo do documento
%\title{Um Exemplo de Artigo em \LaTeX}
\author{Nome-Aluno1\\ (numero-al1) \and Nome-Aluno2\\ (numero-al2)
         \and Nome-Aluno3\\ (numero-al3)
       } %autores do documento
\date{\today} %data

\begin{document} % corpo do documento
\maketitle % apresentar titulo, autor e data

\begin{abstract}  % resumo do documento
Isto é um resumo do relatório de \pe focando o contexto do trabalho (muito sucinto),
os objectivos concretos e os resultados atingidos.\\
Algum texto curto mas que entusiasme à leitura do relatório de \pe.
\end{abstract}

\tableofcontents % Insere a tabela de indice
%\listoffigures % Insere a tabela de indice figuras
%\listoftables % Insere a tabela de indice tabelas

\chapter{Introdução} \label{chap:intro} %referência cruzada

Neste documento começa-se por mostrar o uso de abreviaturas ou macros com parametros acima definidas.\\
\super{Pedro Rangel Henriques}
\titulo{Um belo Projeto}
\area{Processamento de Linguagens}

Vamos agora indicar o conteúdo habitual da introdução de qualquer relatório.
\begin{description}  % Item com descrição
  \item[Enquadramento]  do tema proposto
  \item[Contexto] do tema que é abordado ao  longo do documento
  \item[Problema] o problema que se quer resolver e o objetivo do projeto relatado
  \item[Objetivo] do relatório
  \item[Resultados ou Contributos] -- pontos a evidenciar
  \item[Estrutura do documento] o que é abordado em cada capitulo.
\end{description}

Na Figura~\ref{fig:layoutDimensions} podemos ver o Layout dos Parâmetros do Formato de Página.

 \begin{figure} %insere figuras
       \centering % Coloca ao centro
       \includegraphics[width=0.5\textwidth]{layoutDimensions}
       % width=0.4\textwidth -- a figura é alterada de forma a que a largura seja 0.5 vezes a largura de um parágrafo normal (textwidth). A altura é calculada de forma a manter a relação altura/largura.
       \caption{Legenda da Figura} \label{fig:layoutDimensions} %legenda da figura
 \end{figure}

Letras gregas são estas $ \alpha \beta \gamma \delta $ que aqui demonstro EM FORMATO INLINE\\

Exemplo simples de fração múltipla \[ \frac{\frac{a * b + c}{4-3}}{3*5} \] simples  EM FORMATO
de DESTAQUE (nova linha)\\\\

Exemplo de LISTAS ENUMERADAS com LETRAS:
\begin{enumerate}[a)]
\item Listar todas as Pessoas identificadas, sem repetições;
\item Listar os Países e Cidades marcadas;
\item Listar as Organizações.\\ % mudar de linha
\end{enumerate}

Mais exemplos de LISTAS ENUMERADAS mas agora com NUMEROS e outras marcas:
\begin{enumerate}
\item Listar todas as Pessoas identificadas, sem repetições;
  \begin{enumerate}[i)]
     \item Listar todas as Pessoas identificadas, sem repetições;
     \item Listar os Países e Cidades marcadas;
     \item Listar as Organizações.\\ % mudar de linha
  \end{enumerate}
\item Listar os Países e Cidades marcadas;
  \begin{enumerate}[2.1)]
     \item Listar todas as Pessoas identificadas, sem repetições;
     \item Listar os Países e Cidades marcadas;
     \item Listar as Organizações.\\ % mudar de linha
  \end{enumerate}
\item Listar as Organizações.
    \begin{enumerate}[1)]
     \item Listar todas as Pessoas identificadas, sem repetições;
     \item Listar os Países e Cidades marcadas;
     \item Listar as Organizações.\\ % mudar de linha
  \end{enumerate}
\end{enumerate}

A mesma enumeração mas no standard DESCRITIVO:
\begin{description}
\item[Etape 1:] Listar todas as Pessoas identificadas, sem repetições;
\item[Etape 2:] Listar os Países e Cidades marcadas;
\item[Etape 3:] Listar as Organizações.\\\\
\end{description}


\textbf{Texto com cores}\\

\textcolor{blue}{texto em azul}\\
\textcolor{red}{texto em vermelho} \\
\textcolor{green}{texto em verde} \\
\textcolor{saddlebrown}{texto em Castanho} \\\\\\

\textbf{Texto destacado}\\

 \textbf{Texto NEGRITO} isto é um texto a NEGRITO \\%texto a negrito
 \textsf{Texto fonte SANS SERIF} isto é um texto SANS SERIF \\ % texto em fonte sans serif dentro de uma expressão matemática.
 \texttt{Texto fonte MÁQUINA}  isto é um texto fonte MÁQUINA\\ %texto na fonte de máquina de escrever dentro de uma expressão matemática.
 \textit{Texto a ITALICO} isto é um texto a ITALICO\\ %texto a italico
 \underline{Texto a SUBLINHADO} isto é um texto a SUBLINHADO \\\\ %texto a sublinhado

\textbf{Tamanhos de LETRA}\\

\large{largas -- large}\\
\Large{maiores -- Large}\\
\LARGE{muito grandes -- LARGE}\\
\huge{enormes --  huge}\\
\Huge{as maiores -- Huge}\\
\tiny{minúscula  -- tiny}\\
\scriptsize{muito pequena -- scriptsize} \\
\footnotesize{bastante pequena  -- footnotesize}\\
\small{pequena -- small}\\
\normalsize{normal -- normalsize}\\\\

\textbf{Exemplo de tabelas}\\
\begin{table}[h!] %Inserir tabela
\begin{center} %colocar a tabela ao centro
\begin{tabular}{ | c | c | c | } %desenha a tabela % {formato das colunas} -- 'l' a coluna e alinhada a esquerda;
                                                                              %'r' a coluna e alinhada a direita;
                                                                              %'c' a coluna e centralizada
  \hline  %desenhar uma linha horizontal de comprimento igual ao da tabela
  cell1 & cell2 & cell3 \\
  \hline
  cell4 & cell5 & cell6 \\
  cell7 & cell8 & cell9 \\
  \hline
\end{tabular}
\end{center}
\caption{Tabela Basica} \label{tab:tabelaBasica}
\end{table}


\begin{table}[h!] %Inserir tabela
\begin{center}
\begin{tabular}{ | m{5em} | p{1cm}| b{1in} | } % 'm': ao meio; 'b': em baixo; 'p': em cima
\hline
cell1 dummy text dummy text dummy text& cell2 & cell3 \\
\hline
cell5 & cell1 dummy text dummy text dummy text & cell6 \\
\hline
cell7 & cell8 & exemplo de uma linha muito extensa \\
\hline
\end{tabular}
\caption{Tabela em formato array } \label{tab:tabelaArray}
%Tabela em formato array é utilizada adaptar o comprimento da linha a largura da coluna
\end{center}
\end{table}


\begin{table}[h!]
\begin{center}
\begin{tabular}{ |l|c|r| }
\hline
col1 & col2 & col3 \\
  \hline
  \multirow{3}{4cm}{Multiple row} & cell2 & cell3 \\ %\multirow{num}{formato}{texto} - este comando faz concatenar num linhas em uma so
                                                    %parâmetro  {4cm} define um comprimento de '4cm' para a primeira coluna e centraliza o texto no meio da célula.

  & cell5 & cell6cell6 \\
  & cell8 & cell9cell9cell9 \\
  \hline
  cell4 & cell5 & cell6 \\
  \hline
\end{tabular}
\end{center}
\caption{Tabela com multilinhas}
\end{table}

\begin{table}[h!] %Inserir tabela
\begin{center}
\begin{tabular}{|l||c|c|c|c|c|}
\hline
\multicolumn{6}{|c|}{\textbf{Horário de Tópicos em Matemática - MAT}}\\
\hline
Horário    &Seg &Ter &Qua &Qui &Sex\\
\hline\hline
13:00-14:40&    &    &    &    & \\
\hline
14:55-16:35&    &    &    &    &TURMA N \\
\hline
16:35-18:15&TURMA N & &TURMA N & & \\
\hline
18:15-19:00& & & & & \\
\hline
\end{tabular}
\caption{Tabela com multicolunas}
\end{center}
\end{table}

\newpage

% colocar omitido um url de um site ou de um documento
Para mais informações sobre LATEX consultar o
 \href{http://www.ptep-online.com/ctan/lshort_port.pdf}{livro}.\\

%Colocar url de um site
 Para mais informações sobre LATEX
 consultar o livro\footnote{acessível em \url{http://www.ptep-online.com/ctan/lshort_port.pdf}}.


\section*{Estrutura do Relatório}
explicar como está organizado o documento, referindo os capítulos existentes em~\cite{deransart:1990}
e a sua articulação explicando o conteúdo de cada um.
No capítulo~\ref{chap:analiseEspecificacao} faz-se uma análise detalhada do problema proposto
de modo a poder-se especificar  as entradas, resultados e formas de transformação.\\
etc. \ldots\\ % reticencias
No capítulo~\ref{concl} termina-se o relatório com uma síntese do que foi dito,
as conclusões e o trabalho futuro



\chapter{Análise e Especificação} \label{chap:analiseEspecificacao} %capitulo e referencia cruzada
\section{Descrição informal do problema} \label{sec:descricaoProblema} %seccao e referencia cruzada
\section{Especificação do Requisitos}
\subsection{Dados} \label{subsec:dados} %subseccao e referencia cruzada
\subsection{Pedidos}
\subsection{Relações}

\chapter{Concepção/desenho da Resolução}
\section{Estruturas de Dados}
\section{Algoritmos}

\chapter{Codificação e Testes}
\section{Alternativas, Decisões e Problemas de Implementação}
\section{Testes realizados e Resultados}
Mostram-se a seguir alguns testes feitos (valores introduzidos) e
os respectivos resultados obtidos:

%\VerbatimInput{teste1.txt}

\chapter{Conclusão} \label{concl}
Síntese do Documento~\cite{araujo:2018,martini:2018}.\\
Estado final do projecto; Análise crítica dos resultados~\cite{Sto77a}.\\
Trabalho futuro.

\appendix % apendice
\chapter{Código do Programa}

Lista-se a seguir o CÓDIGO \antlr~\cite{antlr:2016} do programa
\darius~\cite{maskin:1985} que foi desenvolvido.
\begin{verbatim}
public class Aula()
  {
    int n, m;
    int max(int a, int b)
      {
       ......
       return(max);
      }
  }
\end{verbatim}

Lista-se a seguir UM TEXTO (COM O COMANDO VERBATIN)
\begin{verbatim}
      aqui deve aparecer o código do programa,
      tal como está formato no ficheiro-fonte "darius.java"
      um pouco de matematica $\$$
      caso indesejável $\varepsilon$
\end{verbatim}


Lista-se a seguir (ver a Listing~\ref{lstExe1} abaixo) UM TEXTO não processado porque fixado COM O COMANDO LISTING 
que embora parecido com o Verbatim é muito mais sofisticado.

\begin{lstlisting}[caption={Exemplo de uma Listagem}, label={lstExe1}]
      ou entao aparecer aqui neste sitio um pouco de matematica $\$$
      como alternativa ao anterior.
      e aqui mais um teste $\varepsilon$
\end{lstlisting}

\newpage

É ainda possível IMPORTAR UMA LISTAGEM DIRECTAMENTE DO FICHEIRO tal como se ilustra na Listing~\ref{lstExe2}.

\lstinputlisting[caption={Exemplo de uma importação}, label={lstExe2}]{listagemImportadaLayout.l} %input de um ficheiro da listagem

%-- Fim do documento -- inserção das referencias bibliográficas

%\bibliographystyle{plain} % [1] Numérico pela ordem de citação ou ordem alfabetica
\bibliographystyle{alpha} % [Hen18] abreviação do apelido e data da publicação
%\bibliographystyle{apalike} % (Araujo, 2018) apelido e data da publicação
                             % --para usar este estilo descomente no inicio o comando \usepackage{apalike}

\bibliography{bibLayout} %input do ficheiro de referencias bibliograficas

\end{document} 