\documentclass[11pt,a4paper]{report}

\usepackage[utf8]{inputenc}
\usepackage[portuges]{babel}
\usepackage{indentfirst}
\usepackage{graphicx}
\usepackage{float}
\usepackage{caption}
\usepackage{subcaption}
\usepackage[T1]{fontenc}
\usepackage{listings}
\usepackage{amsmath}
\usepackage{mathtools}
\renewcommand{\familydefault}{\sfdefault}

% packages que adicionei do stor
\usepackage{xspace}
\setlength{\oddsidemargin}{-1cm}
\setlength{\textwidth}{18cm}
\setlength{\headsep}{-1cm}
\setlength{\textheight}{23cm}

\title{Processamento de Linguagens (3º ano de Curso)\\
	\textbf{Trabalho Prático Nº1}\\ Relatório de Desenvolvimento}
\author{Diogo Braga\\ A82547 \and João Silva\\ A82005 \and Ricardo Caçador\\ A81064}
\date{\today}

\begin{document}

\maketitle

\begin{abstract}
	Neste relatório é apresentada a resolução de um exercício referente ao TP1, que tem como objetivos a utilização de Expressões Regulares para descrição de padrões de frases, e a utilização do Flex para gerar filtros de texto em C.
Outro objetivo será ainda desenvolver, apartir de ERs, sistemática e automaticamente Processadores de Linguagens Regulares, que filtrem ou transformem textos com base no conceito de regras de produção Condição-Ação.
\end{abstract}

\tableofcontents

\newpage

\chapter{Introdução}
\label{chap:intro}

Seguindo a fórmula \emph{exe= (N\_Alu\% 7) + 1} e o número de aluno mais baixo presente no nosso grupo (81064), o enunciado correspondente é o \textbf{5 - WikiQuotes: provérbios}.

Este enunciado apresenta-nos um tema relacionado com várias citações de vários autores, vários provérbios de línguas diferentes, e ainda provérbios que são variações dos originais e alguns que são adulterações dos originais. Este tipo de material está armazenado num ficheiro \emph{XML}, onde se encontra muitas vezes organizado e outras vezes não tão bem organizado.

Ao longo deste trabalho produzimos essencialmente 3 filtros cada um com um objetivo bem delimitado. O primeiro filtro atua sobre as citações contidas em páginas de autor, o segundo filtro retém todas as citações cujo título da página comece com "Provérbios" e, por último, o terceiro filtro procura um padrão de provérbios que estejam adulterados. Neste último caso optamos também por filtrar os provérbios que são variações. De referir ainda que ligamos os provérbios adulterados e os que são variações ao respetivo provérbio original. O pergunta 4 do enunciado, o grupo optou por não a fazer separadamente dos outros filtros pelos que cada filtro apresenta estatísticas sobre o seu trabalho.

Com este relatório pretendemos apresentar as nossas opções, algoritmos desenvolvidos e ainda estruturas utilizadas para a realização de cada filtro. Pretendemos também apoiar aquelas que foram as nossas soluções, com conhecimento obtido nas aulas teóricas, relembrando por exemplo o caso dos autómatos.

Para uma melhor visão do que irá ser abordado neste relatório deixamos uma breve descrição daquilo que foi feito. No segundo capítulo foi feita uma análise informal e uma especificação dos requisitos deste projeto. No terceiro capítulo foi realizado o desenho da conceção no qual estão envolvidos os algoritmos e estruturas de dados usados. No quarto capítulo mostramos alguns exemplos de implementações e vários resultados de testes realizados. Por último no capítulo 5 fazemos uma retrospetiva do trabalho realizado e concluímos.



\chapter{Análise e Especificação}
\label{chap:analise}

\section{Descrição informal do problema}

\section{Especificação do Requisitos}
\subsection{Lista de Citações}


\subsection{Lista de Provérbios}

Nesta segunda alínea do exercício, é requerido um filtro que crie uma lista de provérbios (citações contidas em páginas cujo título começa por "Provérbios").


\subsection{Lista de Provérbios Adulterados}


\chapter{Concepção/desenho da Resolução}
\label{chap:concepcao}

\section{Estruturas de Dados}
\subsection{Lista de Citações}


\section{Algoritmos}
\subsection{Lista de Citações}


\subsection{Lista de Provérbios}


\subsection{Lista de Provérbios Adulterados}


\chapter{Codificação e Testes}
\label{chap:codificacao}

\section{Alternativas, Decisões e Problemas de Implementação}
\subsection{Lista de Citações}


\subsection{Lista de Provérbios}


\subsection{Lista de Provérbios Adulterados}


\section{Testes realizados e Resultados}
\subsection{Lista de Citações}


\subsection{Lista de Provérbios}


\subsection{Lista de Provérbios Adulterados}


\chapter{Conclusão}
\label{chap:concl}


\end{document}
